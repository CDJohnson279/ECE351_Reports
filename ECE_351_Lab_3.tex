%%%%%%%%%%%%%%%%%%%%%%%%%%%%%%%%%%%%%%%%%%%%%%%%%%%%%%%%%%%%%%%%
%                                                              %
% Dean Johnson                                                 %
% ECE351                                                       %
% Lab 3                                                        %
% 1/25/2022                                                    %
%                                                              %
%%%%%%%%%%%%%%%%%%%%%%%%%%%%%%%%%%%%%%%%%%%%%%%%%%%%%%%%%%%%%%%%

%%%%%%%%%%%%%%%%%%%%%%%%%%%%%%%%%%%%%%%%%%%
%%% DOCUMENT PREAMBLE %%%
\documentclass[12pt]{report}
\usepackage[english]{babel}
%\usepackage{natbib}
\usepackage{url}
\usepackage[utf8x]{inputenc}
\usepackage{amsmath}
\usepackage[utf8]{inputenc}
\usepackage{graphicx}
\graphicspath{{images/}}
\usepackage{parskip}
\usepackage{fancyhdr}
\usepackage{vmargin}
\usepackage{listings}
\usepackage{hyperref}
\usepackage{xcolor}

\definecolor{codegreen}{rgb}{0,0.6,0}
\definecolor{codegray}{rgb}{0.5,0.5,0.5}
\definecolor{codeblue}{rgb}{0,0,0.95}
\definecolor{backcolour}{rgb}{0.95,0.95,0.92}

\lstdefinestyle{mystyle}{
    backgroundcolor=\color{backcolour},   
    commentstyle=\color{codegreen},
    keywordstyle=\color{codeblue},
    numberstyle=\tiny\color{codegray},
    stringstyle=\color{codegreen},
    basicstyle=\ttfamily\footnotesize,
    breakatwhitespace=false,         
    breaklines=true,                 
    captionpos=b,                    
    keepspaces=true,                 
    numbers=left,                    
    numbersep=5pt,                  
    showspaces=false,                
    showstringspaces=false,
    showtabs=false,                  
    tabsize=2
}
 
\lstset{style=mystyle}

\setmarginsrb{3 cm}{2.5 cm}{3 cm}{2.5 cm}{1 cm}{1.5 cm}{1 cm}{1.5 cm}

\title{Lab 3}								
% Title
\author{ Dean Johnson}						
% Author
\date{2/1/2022}
% Date

\makeatletter
\let\thetitle\@title
\let\theauthor\@author
\let\thedate\@date
\makeatother

\pagestyle{fancy}
\fancyhf{}
\rhead{\theauthor}
\lhead{\thetitle}
\cfoot{\thepage}
%%%%%%%%%%%%%%%%%%%%%%%%%%%%%%%%%%%%%%%%%%%%
\begin{document}

%%%%%%%%%%%%%%%%%%%%%%%%%%%%%%%%%%%%%%%%%%%%%%%%%%%%%%%%%%%%%%%%%%%%%%%%%%%%%%%%%%%%%%%%%

\begin{titlepage}
	\centering
    \vspace*{0.5 cm}
   % \includegraphics[scale = 0.075]{bsulogo.png}\\[1.0 cm]	% University Logo
\begin{center}    \textsc{\Large   ECE 351 - Section \#53 }\\[2.0 cm]	\end{center}% University Name
	\textsc{\Large Discrete Convolution  }\\[0.5 cm]				% Course Code
	\rule{\linewidth}{0.2 mm} \\[0.4 cm]
	{ \huge \bfseries \thetitle}\\
	\rule{\linewidth}{0.2 mm} \\[1.5 cm]
	
	\begin{minipage}{0.4\textwidth}
		\begin{flushleft} \large
		%	\emph{Submitted To:}\\
		%	Name\\
          % Affiliation\\
           %contact info\\
			\end{flushleft}
			\end{minipage}~
			\begin{minipage}{0.4\textwidth}
            
			\begin{flushright} \large
			\emph{Submitted By :} \\
			Dean Johnson  
		\end{flushright}
           
	\end{minipage}\\[2 cm]
	
%	\includegraphics[scale = 0.5]{PICMathLogo.png}
    
    
    
    
	
\end{titlepage}

%%%%%%%%%%%%%%%%%%%%%%%%%%%%%%%%%%%%%%%%%%%%%%%%%%%%%%%%%%%%%%%%%%%%%%%%%%%%%%%%%%%%%%%%%

\tableofcontents
\pagebreak

%%%%%%%%%%%%%%%%%%%%%%%%%%%%%%%%%%%%%%%%%%%%%%%%%%%%%%%%%%%%%%%%%%%%%%%%%%%%%%%%%%%%%%%%%
\renewcommand{\thesection}{\arabic{section}}
\section{Introduction}
 The purpose of this lab is to better acquaint ourselves with convolution and its properties within Python. This relates the topics currently discussed in class with the computational properties in lab and allows for a broader understanding of convolution. 
  The complete code and report can be found at the following link: \href{https://github.com/CDJohnson279}{Github Lab 3}. 

\section{Equations}

The following three equations are to be created and plotted in Task 1 and 2 of Part 1, from the previously defined step and ramp functions. The graphs are of the following equations are shown in the \textbf{Results} and the code in the \textbf{Methodology} section. These equations will be used to create the convolutions that will be seen in Part 2. 
\newline 

\textbf{Equations used for Part 1.}\newline 
$f_1(t) = u(t − 2) − u(t − 9)$\newline
$f_2(t) = e^-^tu(t)$\newline 
$f_3(t) = r(t − 2) [u(t − 2) − u(t − 3)] + r(4 − t) [u(t − 3) − u(t − 4)]$
 

\section{Methodology}
Below in the following listing is the code for Task 1 of Part 1, the three functions are easily created since the step and ramp functions were made in the previous lab. They are redefined above the following statements. 

\begin{lstlisting}[language=Python]
######## Part 1 #########

#Task 1 

#f1(t)=u(t-2)-u(t-9)
t = np . arange (0 , 20 + steps , steps )
def f1(t):
    y=step(t-2)-step(t-9)
    return y 
a = f1(t)

#f2(t) =(e^-t)*u(t)
def f2(t):
    y=np.exp(-t)*step(t)
    return y 
b=f2(t) 

#f3(t) = r(t-2)[u(t-2)-u(t-3)]+r(4-t)[u(t-3)-u(t-4)]
def f3(t):
    y = ramp(t-2)*(step(t-2)-step(t-2))+ramp(4-t)*(step(t-3)-step(t-4)) 
    return y 
c=f3(t)
\end{lstlisting}

\newline
For Task 1 of Part 2 we create a user defined function to perform the convolution that takes two functions as inputs and returns the convolution. This function is shown in the listing below. 
\begin{lstlisting}[language=Python]
###### Part 2 ####
#Task 1 - user defined convolution function

def convolve(f,h):
    len_f = np.size(f) #size of forcing function
    len_h = np.size(h) #size of impulse response
    
    #since functions share a commmon start the total length is 1 les than the sum
    sum = np.zeros(len_f+len_h - 1)
    for i in np.arange(len_f):        #at given time in forcing function
        for j in np.arange(len_h):    #at same time in impulse function
            sum[i+j] = sum[i+j] + f[i]*h[j]  #the total sum of previous time with the current time
    return sum
\end{lstlisting}



\newpage 
\section{Results}
From the equations we are given and the steps in Part 1 we are given the following plots in Figure 1. 

\begin{figure}[htp]
    \centering
    \includegraphics[width=16cm]{Part 1.png}
    \caption{Plotted Equations from Part 1}
    \label{fig:Part 1.png}
\end{figure}

Figures 2 through 4 represent Task 2 - Plot the convolution of f1 and f2, Task 3 - Plot the convolution of f2 and f3, and Task 4 - Plot the convolution of f1 and f3. Each figure has the user-defined convolution on the left in \textcolor{blue}{blue} and the verified convolution using the $scipy.signal.convolve$ on the right in \textcolor{red}{red}. 

\begin{figure}[htp]
    \centering
    \includegraphics[width=16cm]{Part2Task2.png}
    \caption{Part 2 Task 2}
    \label{fig:Part 2 Task 2}
\end{figure}

\begin{figure}[htp]
    \centering
    \includegraphics[width=16cm]{Part2Task3.png}
    \caption{Part 2 Task 3}
    \label{fig:Part 2 Task 3}
\end{figure}
\newpage
\begin{figure}[htp]
    \centering
    \includegraphics[width=16cm]{Part2Task4.png}
    \caption{Part 2 Task 4}
    \label{fig:Part 2 Task 4}
\end{figure}

\section{Questions}
\textbf{1.} I worked alone on this lab, I did confer with classmates outside of lab in order to infer on the general procedure to create the convolution function, however it mostly consisted of personal trial and error until a successful function was verified. \newline
\newline
\textbf{2.} The most difficult part for me was creating the convolution function, initial I had to think about how to take the area of a time dependent function and using a system of for loops that move with the position of each function I succeeded in replicating the $scipy.signal.convolve()$'s graph for the convolution of each function. I would also say that figuring out how to create multiple plots to a single figure also proved to be a great struggle to me and in this lab I was able to solve this issue.\newline
\newline
\textbf{3.} I initially approached this lab with a analytical mindset however after failing to replicate the proper convolution I then changed my mindset to an analytical viewpoint, thinking of the convolution purely from what I could complete using code. 

\section{Conclusion}
Overall this lab proved to be very helpful in furthering my understanding of the convolution integral. I think in breaking it down to a step by step analysis of the forcing function and impulse response at a given time and calculating the area of the two in addition to all the previous area, this allowed me to grasp the concept of convolution. Along with furthering the process of convolution the lab also opened up new doors in using Python and Latex. 


\newpage



\end{document}

