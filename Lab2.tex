%%%%%%%%%%%%%%%%%%%%%%%%%%%%%%%%%%%%%%%%%%%%%%%%%%%%%%%%%%%%%%%%
%                                                              %
% Dean Johnson                                                 %
% ECE351                                                       %
% Lab 1                                                        %
% 1/25/2022                                                    %
%                                                              %
%%%%%%%%%%%%%%%%%%%%%%%%%%%%%%%%%%%%%%%%%%%%%%%%%%%%%%%%%%%%%%%%

%%%%%%%%%%%%%%%%%%%%%%%%%%%%%%%%%%%%%%%%%%%
%%% DOCUMENT PREAMBLE %%%
\documentclass[12pt]{report}
\usepackage[english]{babel}
%\usepackage{natbib}
\usepackage{url}
\usepackage[utf8x]{inputenc}
\usepackage{amsmath}
\usepackage[utf8]{inputenc}
\usepackage{graphicx}
\graphicspath{{images/}}
\usepackage{parskip}
\usepackage{fancyhdr}
\usepackage{vmargin}
\usepackage{listings}
\usepackage{hyperref}
\usepackage{xcolor}

\definecolor{codegreen}{rgb}{0,0.6,0}
\definecolor{codegray}{rgb}{0.5,0.5,0.5}
\definecolor{codeblue}{rgb}{0,0,0.95}
\definecolor{backcolour}{rgb}{0.95,0.95,0.92}

\lstdefinestyle{mystyle}{
    backgroundcolor=\color{backcolour},   
    commentstyle=\color{codegreen},
    keywordstyle=\color{codeblue},
    numberstyle=\tiny\color{codegray},
    stringstyle=\color{codegreen},
    basicstyle=\ttfamily\footnotesize,
    breakatwhitespace=false,         
    breaklines=true,                 
    captionpos=b,                    
    keepspaces=true,                 
    numbers=left,                    
    numbersep=5pt,                  
    showspaces=false,                
    showstringspaces=false,
    showtabs=false,                  
    tabsize=2
}
 
\lstset{style=mystyle}

\setmarginsrb{3 cm}{2.5 cm}{3 cm}{2.5 cm}{1 cm}{1.5 cm}{1 cm}{1.5 cm}

\title{1}								
% Title
\author{ Dean Johnson}						
% Author
\date{1/26/2022}
% Date

\makeatletter
\let\thetitle\@title
\let\theauthor\@author
\let\thedate\@date
\makeatother

\pagestyle{fancy}
\fancyhf{}
\rhead{\theauthor}
\lhead{\thetitle}
\cfoot{\thepage}
%%%%%%%%%%%%%%%%%%%%%%%%%%%%%%%%%%%%%%%%%%%%
\begin{document}

%%%%%%%%%%%%%%%%%%%%%%%%%%%%%%%%%%%%%%%%%%%%%%%%%%%%%%%%%%%%%%%%%%%%%%%%%%%%%%%%%%%%%%%%%

\begin{titlepage}
	\centering
    \vspace*{0.5 cm}
   % \includegraphics[scale = 0.075]{bsulogo.png}\\[1.0 cm]	% University Logo
\begin{center}    \textsc{\Large   ECE 351 - Section \#53 }\\[2.0 cm]	\end{center}% University Name
	\textsc{\Large Lab 2 - User Defined Functions  }\\[0.5 cm]				% Course Code
	\rule{\linewidth}{0.2 mm} \\[0.4 cm]
	{ \huge \bfseries \thetitle}\\
	\rule{\linewidth}{0.2 mm} \\[1.5 cm]
	
	\begin{minipage}{0.4\textwidth}
		\begin{flushleft} \large
		%	\emph{Submitted To:}\\
		%	Name\\
          % Affiliation\\
           %contact info\\
			\end{flushleft}
			\end{minipage}~
			\begin{minipage}{0.4\textwidth}
            
			\begin{flushright} \large
			\emph{Submitted By :} \\
			Dean Johnson  
		\end{flushright}
           
	\end{minipage}\\[2 cm]
	
%	\includegraphics[scale = 0.5]{PICMathLogo.png}
    
    
    
    
	
\end{titlepage}

%%%%%%%%%%%%%%%%%%%%%%%%%%%%%%%%%%%%%%%%%%%%%%%%%%%%%%%%%%%%%%%%%%%%%%%%%%%%%%%%%%%%%%%%%

\tableofcontents
\pagebreak

%%%%%%%%%%%%%%%%%%%%%%%%%%%%%%%%%%%%%%%%%%%%%%%%%%%%%%%%%%%%%%%%%%%%%%%%%%%%%%%%%%%%%%%%%
\renewcommand{\thesection}{\arabic{section}}
\section{Introduction}
 

The following report documents the procedures done in Lab 2 of ECE351 - Signals and Systems. In this lab several user defined functions are created. The first part entails recreating a function provided in the procedure, the second is to use a couple of user defined functions to generate a plot of a full plot. The final step is to use or function from part two and use phase shifts and time scaling and lastly a derivative function to generate plots for each step. The github link for this Lab is:  \href{https://github.com/CDJohnson279}{Github Lab 2}. 

\section{Part 1}

Within the documentation provided two sample graphs of a sin function are shown. It is then asked of us to create a user defined function that shows a cos function. This is done as shown below in Figure 1.

\begin{figure}[htp]
    \centering
    \includegraphics[width=4cm]{Part 1 plot.png}
    \caption{ Plot of cosine function}
    \label{fig:Part 1 plot.png}
\end{figure}

\section{Part 2}

For Part 2 we are asked to equation derive an equation for the plot provided. The plot is a series of ramp and step functions the derived equation is:
$y = r(t) - r(t-3) + 5*u(t-3) - 2*u(t-6) - 2*r(t-6) + 2*r(t-10)$
The resulting plot is shown in Figure 2. 
\begin{figure}[htp]
    \centering
    \includegraphics[width=4cm]{Plot for task 3 of lab 2.png}
    \caption{Initial plot}
    \label{fig:initial plot.png}
\end{figure}
The code I derived is shown in the code below. 

\begin{lstlisting}[language=Python]
y = ramp(t) - ramp(t-3) + 5*step(t-3) - 2*step(t-6) - 2*ramp(t-6) + 2*ramp(t-10)
\end{lstlisting}



\section{Part 3}

Part 3 instructs for 5 tasks to be performed the first is to apply a time reversal plot. This figure is included as Figure 3.

\begin{figure}[htp]
    \centering
    \includegraphics[width=4cm]{time reversal.png}
    \caption{Time Reversal Plot}
    \label{fig:time revarsal.png}
\end{figure}

The second task instructs for time shifts to be applied f(t-4) and f(-t-4). The plots are shown below in Figure 4 and 5 respectively. 


\begin{figure}[htp]
    \centering
    \includegraphics[width=4cm]{timeshiftf(t-4).png}
    \caption{Time Shift Plot f(t-4)}
    \label{fig:timeshift.png}
\end{figure}

\begin{figure}[htp]
    \centering
    \includegraphics[width=4cm]{timeshift2.png}
    \caption{Time Shift Plot f(-t-4)}
    \label{fig:timeshiftw.png}
\end{figure}

Task 4 asks for a hand drawn plot of the derivative of the determined function. This is shown in Figure 6.
\begin{figure}[htp]
    \centering
    \includegraphics[width=4cm]{20220126_120108.jpg}
    \caption{Hand Drawn Derivative Plot}
    \label{fig:diffplot.png}
\end{figure}

Lastly task 5 asks us to impliment the numpy.diff() function this function is researched from the doucmentation provided and with the given the following final chart is provided in the last figure below. 

\begin{figure}[htp]
    \centering
    \includegraphics[width=4cm]{derivative plot.png}
    \caption{Derivative Plot}
    \label{fig:numpydiffplot.png}
\end{figure}

\section{Questions}
1. Question 1 asks if the two plots for the derivatives are the same, in my case they were not the question then asks if they could be the same, I imagine if the proper derivatives were calculated that they could perhaps be matched. \newline.
2. Question 2 asks if the correlation between changes with step size and why this would happen. The two become more a like this is since the larger the sampling size the more accurate the results. 

\section{Conclusion}

To conclude the findings of this report Python proves to be a very valuable tool in plotting step and ramp functions as well as for plotting any functions and determining user defined functions and implementing them for various processes. 

\newpage



\end{document}

