%%%%%%%%%%%%%%%%%%%%%%%%%%%%%%%%%%%%%%%%%%%%%%%%%%%%%%%%%%%%%%%%
%                                                              %
% Dean Johnson                                                 %
% ECE351                                                       %
% Lab 8                                                     %
% 3/8/2022                                                    %
%                                                              %
%%%%%%%%%%%%%%%%%%%%%%%%%%%%%%%%%%%%%%%%%%%%%%%%%%%%%%%%%%%%%%%%

%%%%%%%%%%%%%%%%%%%%%%%%%%%%%%%%%%%%%%%%%%%
%%% DOCUMENT PREAMBLE %%%
\documentclass[12pt]{report}
\usepackage[english]{babel}
%\usepackage{natbib}
\usepackage{url}
\usepackage[utf8x]{inputenc}
\usepackage{amsmath}
\usepackage[utf8]{inputenc}
\usepackage{graphicx}
\graphicspath{{images/}}
\usepackage{parskip}
\usepackage{fancyhdr}
\usepackage{vmargin}
\usepackage{listings}
\usepackage{hyperref}
\usepackage{xcolor}

\definecolor{codegreen}{rgb}{0,0.6,0}
\definecolor{codegray}{rgb}{0.5,0.5,0.5}
\definecolor{codeblue}{rgb}{0,0,0.95}
\definecolor{backcolour}{rgb}{0.95,0.95,0.92}

\lstdefinestyle{mystyle}{
    backgroundcolor=\color{backcolour},   
    commentstyle=\color{codegreen},
    keywordstyle=\color{codeblue},
    numberstyle=\tiny\color{codegray},
    stringstyle=\color{codegreen},
    basicstyle=\ttfamily\footnotesize,
    breakatwhitespace=false,         
    breaklines=true,                 
    captionpos=b,                    
    keepspaces=true,                 
    numbers=left,                    
    numbersep=5pt,                  
    showspaces=false,                
    showstringspaces=false,
    showtabs=false,                  
    tabsize=2
}
 
\lstset{style=mystyle}

\setmarginsrb{3 cm}{2.5 cm}{3 cm}{2.5 cm}{1 cm}{1.5 cm}{1 cm}{1.5 cm}

\title{Lab 8}								
% Title
\author{ Dean Johnson}						
% Author
\date{3/1/2022}
% Date

\makeatletter
\let\thetitle\@title
\let\theauthor\@author
\let\thedate\@date
\makeatother

\pagestyle{fancy}
\fancyhf{}
\rhead{\theauthor}
\lhead{\thetitle}
\cfoot{\thepage}
%%%%%%%%%%%%%%%%%%%%%%%%%%%%%%%%%%%%%%%%%%%%
\begin{document}

%%%%%%%%%%%%%%%%%%%%%%%%%%%%%%%%%%%%%%%%%%%%%%%%%%%%%%%%%%%%%%%%%%%%%%%%%%%%%%%%%%%%%%%%%

\begin{titlepage}
	\centering
    \vspace*{0.5 cm}
   % \includegraphics[scale = 0.075]{bsulogo.png}\\[1.0 cm]	% University Logo
\begin{center}    \textsc{\Large   ECE 351 - Section \#53 }\\[2.0 cm]	\end{center}% University Name
	\textsc{\Large Fourier Series Approximation of a Square Wave}\\[0.5 cm]				% Course Code
	\rule{\linewidth}{0.2 mm} \\[0.4 cm]
	{ \huge \bfseries \thetitle}\\
	\rule{\linewidth}{0.2 mm} \\[1.5 cm]
	
	\begin{minipage}{0.4\textwidth}
		\begin{flushleft} \large
		%	\emph{Submitted To:}\\
		%	Name\\
          % Affiliation\\
           %contact info\\
			\end{flushleft}
			\end{minipage}~
			\begin{minipage}{0.4\textwidth}
            
			\begin{flushright} \large
			\emph{Submitted By :} \\
			Dean Johnson  
		\end{flushright}
           
	\end{minipage}\\[2 cm]
	
%	\includegraphics[scale = 0.5]{PICMathLogo.png}
    
    
    
    
	
\end{titlepage}

%%%%%%%%%%%%%%%%%%%%%%%%%%%%%%%%%%%%%%%%%%%%%%%%%%%%%%%%%%%%%%%%%%%%%%%%%%%%%%%%%%%%%%%%%

\tableofcontents
\pagebreak

%%%%%%%%%%%%%%%%%%%%%%%%%%%%%%%%%%%%%%%%%%%%%%%%%%%%%%%%%%%%%%%%%%%%%%%%%%%%%%%%%%%%%%%%%
\renewcommand{\thesection}{\arabic{section}}
\section{Introduction}
The purpose of this lab is to utilize Fourier Series to approximate periodic time-domain signals. This is done in this lab by plotting a square wave using a series of summations that goes closer to the actual square wave the greater the number of summations. 
\newline \newline The GitHub link: \href{https://github.com/CDJohnson279}{Github Lab 8}. 

\section{Methodology}
Prior to the lab  the values for ak, bk and x(t) are determined by hand for the prelab these are included below in the \textbf{Equations} section. The derived expressions for bk, and ak are inputed into Spyder and using python the numerical value for a0,b0,a1,b1,a2,b2,a3,and b3 are solved for these results are included below in the \textbf{Results} section. Using the Fourier series approximation for N = {1,3,15,50,50,1500} with T set to 8s with time from 0 to 20 seconds. These approximations for x(t) are then plotted as seen in the \textbf{Results} section.  

\section{Equations}

\begin{figure}[htp]
    \includegraphics[width=10cm]{Prelab.png}
    \label{fig:Prelab Equations.png}
\end{figure}\newline


\newpage

\section{Results}
\textbf{Part 1}\newline
Task 1: Printed results for k=1, k=2, k=3
\begin{figure}[htp]
    \centering
    \includegraphics[width=10cm]{Results.png}
    \caption{a[0:3] and b[0:3]}
    \label{fig:results Equations.png}
\end{figure}
\newpage
Task 2: Plot of Square Wave Approximations using Fourier Series
\begin{figure}[htp]
    \centering
    \includegraphics[width=10cm]{Plots.png}
    \caption{Square Wave Approximation Plots for N = {1,3,15,50,150,1500}}
    \label{fig:results Equations.png}
\end{figure}
\newpage
\section{Questions}
\textbf{1. Is x(t) an even or an odd function? Explain why.}\newline
x(t) is an even function since by definition x(-t) = -x(t).\newline\newline
\textbf{2. Based on your results from Task 1, what do you expect the values of a2, a3, . . . , an to be?
Why?} \newline
From the prelab and Task 1 we can see that any iteration of ak will be equal to 0. \newline\newline
\textbf{3. How does the approximation of the square wave change as the value of N increases? In what
way does the Fourier series struggle to approximate the square wave?}\newline 
The approximation grows closer to the actual square wave as the value of N increases, the Fourier series struggles to approximate the square wave since there will always be a sinusoidal component to the wave, however the larger the value of the approximation the more accurate it becomes. \newline\newline
\textbf{4. What is occurring mathematically in the Fourier series summation as the value of N increases?}\newline
As the value of N increases, the number of Fourier iterations also increases thus increasing the accuracy of the series. 

\section{Conclusion}
Overall this lab proved to show the usability of Fourier Series, in this case we can see that using multiple summations of a Fourier series allows us to create a square wave with varying accuracy, as the number of summations increases the greater the accuracy is. Using the python software and plots we can easily visualize this concept. 
\newpage


\end{document}

