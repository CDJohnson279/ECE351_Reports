%%%%%%%%%%%%%%%%%%%%%%%%%%%%%%%%%%%%%%%%%%%%%%%%%%%%%%%%%%%%%%%%
%                                                              %
% Dean Johnson                                                 %
% ECE351                                                       %
% Lab 10                                                    %
% 3/29/2022                                                    %
%                                                              %
%%%%%%%%%%%%%%%%%%%%%%%%%%%%%%%%%%%%%%%%%%%%%%%%%%%%%%%%%%%%%%%%

%%%%%%%%%%%%%%%%%%%%%%%%%%%%%%%%%%%%%%%%%%%
%%% DOCUMENT PREAMBLE %%%
\documentclass[12pt]{report}
\usepackage[english]{babel}
%\usepackage{natbib}
\usepackage{url}
\usepackage[utf8x]{inputenc}
\usepackage{amsmath}
\usepackage[utf8]{inputenc}
\usepackage{graphicx}
\graphicspath{{images/}}
\usepackage{parskip}
\usepackage{fancyhdr}
\usepackage{vmargin}
\usepackage{listings}
\usepackage{hyperref}
\usepackage{xcolor}
\usepackage{float}

\definecolor{codegreen}{rgb}{0,0.6,0}
\definecolor{codegray}{rgb}{0.5,0.5,0.5}
\definecolor{codeblue}{rgb}{0,0,0.95}
\definecolor{backcolour}{rgb}{0.95,0.95,0.92}

\lstdefinestyle{mystyle}{
    backgroundcolor=\color{backcolour},   
    commentstyle=\color{codegreen},
    keywordstyle=\color{codeblue},
    numberstyle=\tiny\color{codegray},
    stringstyle=\color{codegreen},
    basicstyle=\ttfamily\footnotesize,
    breakatwhitespace=false,         
    breaklines=true,                 
    captionpos=b,                    
    keepspaces=true,                 
    numbers=left,                    
    numbersep=5pt,                  
    showspaces=false,                
    showstringspaces=false,
    showtabs=false,                  
    tabsize=2
}
 
\lstset{style=mystyle}

\setmarginsrb{3 cm}{2.5 cm}{3 cm}{2.5 cm}{1 cm}{1.5 cm}{1 cm}{1.5 cm}

\title{Lab 10}								
% Title
\author{ Dean Johnson}						
% Author
\date{3/29/2022}
% Date

\makeatletter
\let\thetitle\@title
\let\theauthor\@author
\let\thedate\@date
\makeatother

\pagestyle{fancy}
\fancyhf{}
\rhead{\theauthor}
\lhead{\thetitle}
\cfoot{\thepage}
%%%%%%%%%%%%%%%%%%%%%%%%%%%%%%%%%%%%%%%%%%%%
\begin{document}

%%%%%%%%%%%%%%%%%%%%%%%%%%%%%%%%%%%%%%%%%%%%%%%%%%%%%%%%%%%%%%%%%%%%%%%%%%%%%%%%%%%%%%%%%

\begin{titlepage}
	\centering
    \vspace*{0.5 cm}
   % \includegraphics[scale = 0.075]{bsulogo.png}\\[1.0 cm]	% University Logo
\begin{center}    \textsc{\Large   ECE 351 - Section \#53 }\\[2.0 cm]	\end{center}% University Name
	\textsc{\Large Frequency Response}\\[0.5 cm]				% Course Code
	\rule{\linewidth}{0.2 mm} \\[0.4 cm]
	{ \huge \bfseries \thetitle}\\
	\rule{\linewidth}{0.2 mm} \\[1.5 cm]
	
	\begin{minipage}{0.4\textwidth}
		\begin{flushleft} \large
		%	\emph{Submitted To:}\\
		%	Name\\
          % Affiliation\\
           %contact info\\
			\end{flushleft}
			\end{minipage}~
			\begin{minipage}{0.4\textwidth}
            
			\begin{flushright} \large
			\emph{Submitted By :} \\
			Dean Johnson  
		\end{flushright}
           
	\end{minipage}\\[2 cm]
	
%	\includegraphics[scale = 0.5]{PICMathLogo.png}
    
    
    
    
	
\end{titlepage}

%%%%%%%%%%%%%%%%%%%%%%%%%%%%%%%%%%%%%%%%%%%%%%%%%%%%%%%%%%%%%%%%%%%%%%%%%%%%%%%%%%%%%%%%%

\tableofcontents
\pagebreak

%%%%%%%%%%%%%%%%%%%%%%%%%%%%%%%%%%%%%%%%%%%%%%%%%%%%%%%%%%%%%%%%%%%%%%%%%%%%%%%%%%%%%%%%%
\renewcommand{\thesection}{\arabic{section}}
\section{Introduction}
The goal of this lab is to use frequency response tools and Bode plots within Python to further understand frequency response. 
\newline \newline The GitHub link: \href{https://github.com/CDJohnson279}{Github Lab 10}. 

\section{Methodology}
For part one we are asked to use the expressions derived in the prelab to create Bode plots for the provided transfer function, for the phase and magnitude. This is done using the matplotlib functions. For the first task the transfer function is plotted on the x-axis on a logarithmic scale. The transfer function is then used within the scipy bode plot function and the phase is adjusted so that it matches the plot created by using the hand derived calculations. The frequency response of the system is then plotted in hertz using the provided code. The plots for the first part are shown in the \textbf{Results} section below. 
Part 2 asks us to plot a signal that is passed through the filter from Part 1 using the bilinear and lfilter functions. These two plots are also included in the \textbf{Results} section below. 
\section{Equations}
\textbf{Prelab Equations} 
\begin{figure}[h!]
    \centering
    \includegraphics[height=8cm]{prelab.png}
    \label{Prelab Equations}
\end{figure}\newline
\textbf{Signal equation for Part 2.}\newline
$x(t) = cos(2\pi *100t) + cos(2\pi *3024t) + sin(2\pi * 50000t)$

\section{Results}
Plots from Part 1.\newline 
\begin{figure}[h!]
    \centering
    \includegraphics[width=9cm]{Part1.1.png}
    \caption{Logarithmic Scale of Phase and Magnitude }
    \label{Task 1}
\end{figure}

\begin{figure}[h!]
    \centering
    \includegraphics[width=10cm]{Part1.2.png}
    \caption{Bode Plots using Scypy.Signal.bode }
    \label{Task 2}
\end{figure}

\begin{figure}[h!]
    \centering
    \includegraphics[width=9cm]{Part1.3.png}
    \caption{Frequency Response of the Transfer Function }
    \label{Task 1}
    \newpage 
\end{figure}
\newpage 
\textbf{Plots from Part 2}
\begin{figure}[H]
    \centering
    \includegraphics[width=9cm]{Part2.1.png}
    \caption{Input Signal Plot }
    \label{Task 1}
\end{figure}
\newpage 
\newpage
\begin{figure}[H]
    \centering
    \includegraphics[width=9cm]{Part2.2.png}
    \caption{Output Signal Plot }
    \label{Task 1}
\end{figure}
 \newpage  

 
\section{Questions}
\textbf{1. Explain how the filter and filtered output in Part 2 makes sense given the Bode plots from
Part 1. Discuss how the filter modifies specific frequency bands, in Hz.}\newline \newline
The output plots are affected by the filter in such manner that the magnitudes and phases are changed at given frequencies with respect to the transfer function. This can be seen by taking the input signal at a given frequency and viewing the value of the phase and amplitude change within the determined bode plots. \newline \newline 
\textbf{2. Discuss the purpose and workings of
scipy.signal.bilinear() and scipy.signal.lfilter().}\newline\newline
The bilinear function takes in the numerator and denominator of a transfer function in the form of an array, and transforms them from the s domain to the z domain. The lfilter function takes, the in the numerator, denominator, of the filter function and an array of an input, and returns the value of the input array through the digital filter. \newline

\textbf{3. What happens if you use a different sampling frequency in scipy.signal.bilinear() than
you used for the time-domain signal?}\newline \newline
When using a low sampling frequency the plot samples more frequently and makes the plot dense, however with a higher sampling frequency the plot is more sparse and easier to interpret. 
\section{Conclusion}
This lab continues to illustrate topics discussed in class, showing how a signal is changed by a transfer function or a filter. This can be visually interpreted by viewing the Bode plots and applying the output accordingly. 
\newpage


\end{document}

