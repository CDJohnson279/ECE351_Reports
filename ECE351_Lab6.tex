%%%%%%%%%%%%%%%%%%%%%%%%%%%%%%%%%%%%%%%%%%%%%%%%%%%%%%%%%%%%%%%%
%                                                              %
% Dean Johnson                                                 %
% ECE351                                                       %
% Lab 6                                                       %
% 2/22/2022                                                    %
%                                                              %
%%%%%%%%%%%%%%%%%%%%%%%%%%%%%%%%%%%%%%%%%%%%%%%%%%%%%%%%%%%%%%%%

%%%%%%%%%%%%%%%%%%%%%%%%%%%%%%%%%%%%%%%%%%%
%%% DOCUMENT PREAMBLE %%%
\documentclass[12pt]{report}
\usepackage[english]{babel}
%\usepackage{natbib}
\usepackage{url}
\usepackage[utf8x]{inputenc}
\usepackage{amsmath}
\usepackage[utf8]{inputenc}
\usepackage{graphicx}
\graphicspath{{images/}}
\usepackage{parskip}
\usepackage{fancyhdr}
\usepackage{vmargin}
\usepackage{listings}
\usepackage{hyperref}
\usepackage{xcolor}

\definecolor{codegreen}{rgb}{0,0.6,0}
\definecolor{codegray}{rgb}{0.5,0.5,0.5}
\definecolor{codeblue}{rgb}{0,0,0.95}
\definecolor{backcolour}{rgb}{0.95,0.95,0.92}

\lstdefinestyle{mystyle}{
    backgroundcolor=\color{backcolour},   
    commentstyle=\color{codegreen},
    keywordstyle=\color{codeblue},
    numberstyle=\tiny\color{codegray},
    stringstyle=\color{codegreen},
    basicstyle=\ttfamily\footnotesize,
    breakatwhitespace=false,         
    breaklines=true,                 
    captionpos=b,                    
    keepspaces=true,                 
    numbers=left,                    
    numbersep=5pt,                  
    showspaces=false,                
    showstringspaces=false,
    showtabs=false,                  
    tabsize=2
}
 
\lstset{style=mystyle}

\setmarginsrb{3 cm}{2.5 cm}{3 cm}{2.5 cm}{1 cm}{1.5 cm}{1 cm}{1.5 cm}

\title{Lab 6}								
% Title
\author{ Dean Johnson}						
% Author
\date{2/22/2022}
% Date

\makeatletter
\let\thetitle\@title
\let\theauthor\@author
\let\thedate\@date
\makeatother

\pagestyle{fancy}
\fancyhf{}
\rhead{\theauthor}
\lhead{\thetitle}
\cfoot{\thepage}
%%%%%%%%%%%%%%%%%%%%%%%%%%%%%%%%%%%%%%%%%%%%
\begin{document}

%%%%%%%%%%%%%%%%%%%%%%%%%%%%%%%%%%%%%%%%%%%%%%%%%%%%%%%%%%%%%%%%%%%%%%%%%%%%%%%%%%%%%%%%%

\begin{titlepage}
	\centering
    \vspace*{0.5 cm}
   % \includegraphics[scale = 0.075]{bsulogo.png}\\[1.0 cm]	% University Logo
\begin{center}    \textsc{\Large   ECE 351 - Section \#53 }\\[2.0 cm]	\end{center}% University Name
	\textsc{\Large Partial Fraction Expansion}\\[0.5 cm]				% Course Code
	\rule{\linewidth}{0.2 mm} \\[0.4 cm]
	{ \huge \bfseries \thetitle}\\
	\rule{\linewidth}{0.2 mm} \\[1.5 cm]
	
	\begin{minipage}{0.4\textwidth}
		\begin{flushleft} \large
		%	\emph{Submitted To:}\\
		%	Name\\
          % Affiliation\\
           %contact info\\
			\end{flushleft}
			\end{minipage}~
			\begin{minipage}{0.4\textwidth}
            
			\begin{flushright} \large
			\emph{Submitted By :} \\
			Dean Johnson  
		\end{flushright}
           
	\end{minipage}\\[2 cm]
	
%	\includegraphics[scale = 0.5]{PICMathLogo.png}
    
    
    
    
	
\end{titlepage}

%%%%%%%%%%%%%%%%%%%%%%%%%%%%%%%%%%%%%%%%%%%%%%%%%%%%%%%%%%%%%%%%%%%%%%%%%%%%%%%%%%%%%%%%%

\tableofcontents
\pagebreak

%%%%%%%%%%%%%%%%%%%%%%%%%%%%%%%%%%%%%%%%%%%%%%%%%%%%%%%%%%%%%%%%%%%%%%%%%%%%%%%%%%%%%%%%%
\renewcommand{\thesection}{\arabic{section}}
\section{Introduction}
 The goal of this lab is to use the built in function in Spyder to solve and perform partial fraction expansion in order to solve complex differential equations using Laplace transforms. \newline \newline The GitHub link: \href{https://github.com/CDJohnson279}{Github Lab 6}. 

\section{Methodology}
Prior to the lab it asked to complete a prelab assignment in this assignment the prelab asks us to find the transfer function of a second order differential equation. then using partial fraction expansion and inverse laplace transforms to find the value of y(t) if the input is a step function. 
For part 1 of the lab we are asked to plot the equation for the step response we derived in the prelab. this shown in Figure 1 of the \textbf{Results} section. Next the step response is plotted using the scipy.signal.step command. This is also shown in Figure 1. below. The partial fraction of the equation for Y(s) is then determined using the scipy.signal.residue function these values are then compared to the values determined in the prelab. The values I determined were 1,10,24 however the values obtained from the scipy function do not resemble these values. 
Part 2 asks us to use the scip.signal.resdue functio to perform partial fraction on a more difficult function, one that would quite difficutl to analyze by hand. The time domain function of this function is determined using the cosine method, this derived function is shown below in the \textbf{Results} section, along with the partial ffraction expansion of the step response. Lastly the H(s) function is plotted and compared vs the scipy.signal.step plot. 

\section{Equations}
The given equation:
$$y
′′(t) + 10y
′
(t) + 24y(t) = x
′′(t) + 6x
′
(t) + 12x(t)$$
The derived transfer function: 
$$H(s) = 1 + \frac{4s+12}{s^2+6s+12}$$
Hand calculated y(t):
$$y(t) = y(t) = (.5+.5e^-^4^t-e^-^6^t)u(t)$$
\section{Results}

\begin{figure}[htp]
    \centering
    \includegraphics[width=16cm]{Part 1.1.png}
    \caption{Plots for Part 1.}
    \label{fig:Part 1.png}
\end{figure}

\begin{figure}[htp]
    \centering
    \includegraphics[width=8cm]{Partials for Part 1.png}
    \caption{Partial fraction results for Part 1}
    \label{fig:Partials for Part 1.png}
\end{figure}
\newpage

\begin{figure}[htp]
    \centering
    \includegraphics[width=16cm]{Part 2.png}
    \caption{Plots for Part 2.}
    \label{fig:Part 1.png}
\end{figure}

\begin{figure}[htp]
    \centering
    \includegraphics[height=4cm]{Partials for Part 2.PNG}
    \caption{Partial fraction results for Part 2.}
    \label{fig:Part 1.png}
\end{figure}
\newpage
\section{Questions}
\textbf{1. For a non-complex pole-residue term, you can still use the cosine method, explain why this works.}\newline
For non complex poles the cosine method still works since inside of the cosine function there will not be any negative bounds therefore the cosine method will still work effectively.
\section{Conclusion}
This lab was effective in illustrating the value of utilizing a computer created function to solve a very complex differential equation and solving for the partial fractions for more complex equations.
\newpage



\end{document}

