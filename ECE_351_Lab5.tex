%%%%%%%%%%%%%%%%%%%%%%%%%%%%%%%%%%%%%%%%%%%%%%%%%%%%%%%%%%%%%%%%
%                                                              %
% Dean Johnson                                                 %
% ECE351                                                       %
% Lab 5                                                        %
% 2/15/2022                                                    %
%                                                              %
%%%%%%%%%%%%%%%%%%%%%%%%%%%%%%%%%%%%%%%%%%%%%%%%%%%%%%%%%%%%%%%%

%%%%%%%%%%%%%%%%%%%%%%%%%%%%%%%%%%%%%%%%%%%
%%% DOCUMENT PREAMBLE %%%
\documentclass[12pt]{report}
\usepackage[english]{babel}
%\usepackage{natbib}
\usepackage{url}
\usepackage[utf8x]{inputenc}
\usepackage{amsmath}
\usepackage[utf8]{inputenc}
\usepackage{graphicx}
\graphicspath{{images/}}
\usepackage{parskip}
\usepackage{fancyhdr}
\usepackage{vmargin}
\usepackage{listings}
\usepackage{hyperref}
\usepackage{xcolor}

\definecolor{codegreen}{rgb}{0,0.6,0}
\definecolor{codegray}{rgb}{0.5,0.5,0.5}
\definecolor{codeblue}{rgb}{0,0,0.95}
\definecolor{backcolour}{rgb}{0.95,0.95,0.92}

\lstdefinestyle{mystyle}{
    backgroundcolor=\color{backcolour},   
    commentstyle=\color{codegreen},
    keywordstyle=\color{codeblue},
    numberstyle=\tiny\color{codegray},
    stringstyle=\color{codegreen},
    basicstyle=\ttfamily\footnotesize,
    breakatwhitespace=false,         
    breaklines=true,                 
    captionpos=b,                    
    keepspaces=true,                 
    numbers=left,                    
    numbersep=5pt,                  
    showspaces=false,                
    showstringspaces=false,
    showtabs=false,                  
    tabsize=2
}
 
\lstset{style=mystyle}

\setmarginsrb{3 cm}{2.5 cm}{3 cm}{2.5 cm}{1 cm}{1.5 cm}{1 cm}{1.5 cm}

\title{Lab 5}								
% Title
\author{ Dean Johnson}						
% Author
\date{2/1/2022}
% Date

\makeatletter
\let\thetitle\@title
\let\theauthor\@author
\let\thedate\@date
\makeatother

\pagestyle{fancy}
\fancyhf{}
\rhead{\theauthor}
\lhead{\thetitle}
\cfoot{\thepage}
%%%%%%%%%%%%%%%%%%%%%%%%%%%%%%%%%%%%%%%%%%%%
\begin{document}

%%%%%%%%%%%%%%%%%%%%%%%%%%%%%%%%%%%%%%%%%%%%%%%%%%%%%%%%%%%%%%%%%%%%%%%%%%%%%%%%%%%%%%%%%

\begin{titlepage}
	\centering
    \vspace*{0.5 cm}
   % \includegraphics[scale = 0.075]{bsulogo.png}\\[1.0 cm]	% University Logo
\begin{center}    \textsc{\Large   ECE 351 - Section \#53 }\\[2.0 cm]	\end{center}% University Name
	\textsc{\Large Step and Impulse Response of a RLC Band Pass Filter }\\[0.5 cm]				% Course Code
	\rule{\linewidth}{0.2 mm} \\[0.4 cm]
	{ \huge \bfseries \thetitle}\\
	\rule{\linewidth}{0.2 mm} \\[1.5 cm]
	
	\begin{minipage}{0.4\textwidth}
		\begin{flushleft} \large
		%	\emph{Submitted To:}\\
		%	Name\\
          % Affiliation\\
           %contact info\\
			\end{flushleft}
			\end{minipage}~
			\begin{minipage}{0.4\textwidth}
            
			\begin{flushright} \large
			\emph{Submitted By :} \\
			Dean Johnson  
		\end{flushright}
           
	\end{minipage}\\[2 cm]
	
%	\includegraphics[scale = 0.5]{PICMathLogo.png}
    
    
    
    
	
\end{titlepage}

%%%%%%%%%%%%%%%%%%%%%%%%%%%%%%%%%%%%%%%%%%%%%%%%%%%%%%%%%%%%%%%%%%%%%%%%%%%%%%%%%%%%%%%%%

\tableofcontents
\pagebreak

%%%%%%%%%%%%%%%%%%%%%%%%%%%%%%%%%%%%%%%%%%%%%%%%%%%%%%%%%%%%%%%%%%%%%%%%%%%%%%%%%%%%%%%%%
\renewcommand{\thesection}{\arabic{section}}
\section{Introduction}
 The purpose of this lab is utilize Laplace transforms to find the time-domain response of impulse and step inputs for an RLC bandpass filter.  \newline \newline The GitHub link: \href{https://github.com/CDJohnson279}{Github Lab 5}. 

\section{Methodology}
Before the lab I prelab is done that shows the RLC Band Pass Filter circuit, to be evaluated. From this circuit and the given values for the capacitor, resistor, and inductor the transfer function and impulse response are calculated by hand. The derived equations are shown below in the \textbf{Equations} section. Part 1 of the lab asks for the hand solved impuls response to be plotted as well as the impulse response using the scipy impulse function, using the transfer function from the prelab. These plots are included in the \textbf{Results} section. 


\section{Equations}
Below are the hand calculated impulse and step response equations from the prelab. 
\newline
Hand Calculated Transfer Function: $$H(s) = (10000s)/(s^2 + s/(10000) + (3.7037*10^8)$$\newline
Hand Calculated Impulse Response: $$h(t) = 10000e^-^5^0^0^0^t*cos(18584t)-2690.5e^-^5^0^0^0^t*sin(18584t)$$
\newpage 
\section{Results}
The figures below are from task 1 and 2 of Part 1.
\begin{figure}[htp]
    \centering
    \includegraphics[width=16cm]{h(t).png}
    \caption{Time Domain Impulse Response Plots}
    \label{fig:Part 1.png}
\end{figure}
\newline\newpage
The next plot is the is from task 1 of Part 2 and is the step response of H(s) performed by the scipy function. 
\begin{figure}[htp]
    \centering
    \includegraphics[width=16cm]{h(s).png}
    \caption{Step response of H(s)}
    \label{fig:Part 1.png}
\end{figure}
\newline
Both the functions approach zero rapidly, the plot in the Part 2 and the plot from Part 1 both have a very similar shape, however the plot from Part 2, using the final value theorem is a much smaller scale. The latter plot also visually appears to be slightly shifted to the right. 
\section{Questions}
1. Explain the result of the Final Value Theorem from Part 2 Task 2 in terms of the physical
circuit components.\newline
The circuit shows that the circuit has a brief rise and drop in gain and then settles out, this is due to the capacitor and inductor both having an initial release of energy and then the system stabilizes. 

\section{Conclusion}
Within this lab analyzing a transfer function in the Laplace domain allows us to determine the step response of a circuit. Then utilizing python and the plotting abilities we can verify our equation using the built in scipy functions. 

\newpage



\end{document}

