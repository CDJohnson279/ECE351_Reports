%%%%%%%%%%%%%%%%%%%%%%%%%%%%%%%%%%%%%%%%%%%%%%%%%%%%%%%%%%%%%%%%
%                                                              %
% Dean Johnson                                                 %
% ECE351                                                       %
% Lab Final                                                   %
% 4/20/2022                                                    %
%                                                              %
%%%%%%%%%%%%%%%%%%%%%%%%%%%%%%%%%%%%%%%%%%%%%%%%%%%%%%%%%%%%%%%%

%%%%%%%%%%%%%%%%%%%%%%%%%%%%%%%%%%%%%%%%%%%
%%% DOCUMENT PREAMBLE %%%
\documentclass[12pt]{report}
\usepackage[english]{babel}
%\usepackage{natbib}
\usepackage{url}
\usepackage[utf8x]{inputenc}
\usepackage{amsmath}
\usepackage[utf8]{inputenc}
\usepackage{graphicx}
\graphicspath{{images/}}
\usepackage{parskip}
\usepackage{fancyhdr}
\usepackage{vmargin}
\usepackage{listings}
\usepackage{hyperref}
\usepackage{xcolor}
\usepackage{url}


\definecolor{codegreen}{rgb}{0,0.6,0}
\definecolor{codegray}{rgb}{0.5,0.5,0.5}
\definecolor{codeblue}{rgb}{0,0,0.95}
\definecolor{backcolour}{rgb}{0.95,0.95,0.92}

\lstdefinestyle{mystyle}{
    backgroundcolor=\color{backcolour},   
    commentstyle=\color{codegreen},
    keywordstyle=\color{codeblue},
    numberstyle=\tiny\color{codegray},
    stringstyle=\color{codegreen},
    basicstyle=\ttfamily\footnotesize,
    breakatwhitespace=false,         
    breaklines=true,                 
    captionpos=b,                    
    keepspaces=true,                 
    numbers=left,                    
    numbersep=5pt,                  
    showspaces=false,                
    showstringspaces=false,
    showtabs=false,                  
    tabsize=2
}
 
\lstset{style=mystyle}

\setmarginsrb{3 cm}{2.5 cm}{3 cm}{2.5 cm}{1 cm}{1.5 cm}{1 cm}{1.5 cm}

\title{Lab 12 - Final Project}								
% Title
\author{ Dean Johnson}						
% Author
\date{4/20/2022}
% Date

\makeatletter
\let\thetitle\@title
\let\theauthor\@author
\let\thedate\@date
\makeatother

\pagestyle{fancy}
\fancyhf{}
\rhead{\theauthor}
\lhead{\thetitle}
\cfoot{\thepage}
%%%%%%%%%%%%%%%%%%%%%%%%%%%%%%%%%%%%%%%%%%%%
\begin{document}

%%%%%%%%%%%%%%%%%%%%%%%%%%%%%%%%%%%%%%%%%%%%%%%%%%%%%%%%%%%%%%%%%%%%%%%%%%%%%%%%%%%%%%%%%

\begin{titlepage}
	\centering
    \vspace*{0.5 cm}
   % \includegraphics[scale = 0.075]{bsulogo.png}\\[1.0 cm]	% University Logo
\begin{center}    \textsc{\Large   ECE 351 - Section \#53 }\\[2.0 cm]	\end{center}% University Name
	\textsc{\Large Filter Design}\\[0.5 cm]				% Course Code
	\rule{\linewidth}{0.2 mm} \\[0.4 cm]
	{ \huge \bfseries \thetitle}\\
	\rule{\linewidth}{0.2 mm} \\[1.5 cm]
	
	\begin{minipage}{0.4\textwidth}
		\begin{flushleft} \large
		%	\emph{Submitted To:}\\
		%	Name\\
          % Affiliation\\
           %contact info\\
			\end{flushleft}
			\end{minipage}~
			\begin{minipage}{0.4\textwidth}
            
			\begin{flushright} \large
			\emph{Submitted By :} \\
			Dean Johnson  
		\end{flushright}
           
	\end{minipage}\\[2 cm]
	
%	\includegraphics[scale = 0.5]{PICMathLogo.png}
    
    
    
    
	
\end{titlepage}

%%%%%%%%%%%%%%%%%%%%%%%%%%%%%%%%%%%%%%%%%%%%%%%%%%%%%%%%%%%%%%%%%%%%%%%%%%%%%%%%%%%%%%%%%

\tableofcontents
\pagebreak

%%%%%%%%%%%%%%%%%%%%%%%%%%%%%%%%%%%%%%%%%%%%%%%%%%%%%%%%%%%%%%%%%%%%%%%%%%%%%%%%%%%%%%%%%
\renewcommand{\thesection}{\arabic{section}}
\section{Introduction}
The goal of this lab is to apply the skills we have learned from previous labs and this class to a practical application.  
\newline \newline The GitHub link: \href{https://github.com/CDJohnson279}{Github Lab 12 - Final Project}. 

\section{Methodology}
To begin this lab a csv file is provided that contains information for a signal that represents a signal from a positioning control system. We are told that the position measurement is contained with in the range of $ 1.8kHz \le \textit{f} \le 2.0kHz$  It is also stated that the system shares a ground with a switching sensor which causes noise at higher frequencies, and that there is also low frequency vibrations from the buildings systems. Both of these high and low frequencies are asked to be filtered out to create a \textit{clean} output signal. Task 1 of this lab asks us to identify the magnitudes and frequencies that correspond to these noisy values. We also plot the current noisy input signal. \newline 
For Task 2 we are asked to design an analog filter to remove the noise and pass only the desired position measurement values. The filter is to attenuate the values of the signal as follows: 
\begin{itemize}
  \item  position measurement information is attenuated by less than -0.3d
  \item low-frequency vibration noise must be attenuated by at least -30dB
  \item switching amplifier noise must be attenuated by at least -21dB
  \item All noise that exists at frequencies greater than 100kHz must be completely attenuated
\end{itemize}
The determined filter circuit is created along with its transfer function both of these are included in the \textbf{Equations} section below. \newline
For Task 3 the Bode plot is generated to show the attenuated magnitudes and phases of the created RLC filter circuit. Individual plots are created for individual sections of frequency i.e a plot for low frequency, high frequency, and the position measurement. These are discussed in detail below. \newline
Lastly for Task 4 the input signal is passed through the created transfer function and the output signal is plotted now filtered. Then using the Fast Fourier Transform function we previously created the original noisy signal is compared to the new clean output signal to verify the filter specifications from Task 2, this is included in the \textbf{Results} section. 
\section{Equations}
\textbf{Transfer Function:}

$H(s) = \frac{\frac{R}{L}s}{s^2 + \frac{R}{L}s + \frac{1}{LC}}$  \newline\newline 
\textbf{Filter Values:}\newline\newline 
$R = 10k\ohm $\newline
$L = 5.26$H\newline
$C = 52.63$nF\newline
$\beta = 200$Hz\newline
$\omega_0 =1900 $Hz\newline \newline 
\textbf{RLC Filter Circuit:} \newline 
\begin{figure}[h!]
    \includegraphics[width=10cm]{circuit.PNG}
    \label{Task 1 - Input Signal}
\end{figure}
\newpage 
\section{Results}
\textbf{Task 1}\newline 
Below is the provided input signal plotted from the csv file. 
\begin{figure}[h!]
    \centering
    \includegraphics[width=10cm]{Noisy input.png}
    \caption{Noisy Input Signal }
    \label{Task 1}
\end{figure}\newpage 

\textbf{Task 2 - Filter Design}\newline
\\The figure below shows the full Bode plot generated from the derived transfer function. 
\\
\begin{figure}[h!]
    \centering
    \includegraphics[width=9cm]{bode full.png}
    \caption{Bode Plot of Transfer Function}
    \label{Task 1}
\end{figure} 
\\Figure 3. is the previous bode plot but from the range of 0hz to 1.8kHz to show all low frequency vibration noise has been attenuated by at least -30dB.
\\
\begin{figure}[h!]
    \centering
    \includegraphics[width=9cm]{Bode low.png}
    \caption{Low-frequency Bode Plot }
    \label{Task 1}
\end{figure}
\newpage 
\\The last Bode plot shows that all noise due to high frequencies has been completely attenuated and that the switching amplifier noise has been attenuated by at least -21dB.
\\
\begin{figure}[h!]
    \centering
    \includegraphics[width=10cm]{bode high.png}
    \caption{High-frequency Bode Plot }
    \label{Task 1}
\end{figure} 
\newpage
\textbf{Fast Fourier Transform Comparison} \newline
For each of the Fourier transform plots below the unfiltered signal is seen on top and the the filtered plot is on the bottom.
\begin{figure}[h!]
    \centering
    \includegraphics[width=10cm]{FFT unflitered.png}
    \includegraphics[width=10cm]{FFT filtered.png}
    \caption{Full Range Fast Fourier }
    \label{Task 1}
\end{figure} 
\\
From the plots of the full range Fourier transform we can see that most of the noise has been filtered out and that only the position measurement remains. 
\begin{figure}[h!]
    \centering
    \includegraphics[width=10cm]{Unfiltered position.png}
    \includegraphics[width=10cm]{Unfiltered position.png}
    \caption{Position Measurement }
    \label{Task 1}
\end{figure} 
\\
The position measurement plots show that the desired position information remains unchanged, or is minimally attenuated. 
\begin{figure}[h!]
    \centering
    \includegraphics[width=10cm]{unfiltered low.png}
    \includegraphics[width=10cm]{filtered low.png}
    \caption{Low Frequency}
    \label{Task 1}
\end{figure}
\\

The low frequency noise is completed filtered out or down to a negligible magnitude. 
\newpage 
\begin{figure}[h!]
    \centering
    \includegraphics[width=10cm]{unfiltered high.png}
    \includegraphics[width=10cm]{Filtered High.png}
    \caption{High Frequency}
    \label{Task 1}
\end{figure}

\\
Lastly the plot above shows that all noise above the position measurement frequency has been attenuated. 
\newpage 
\section{Questions}
\textbf{Earlier this semester, you were asked what you personally wanted to get
out of taking this course. Do you feel like that personal goal was met?
Why or why not?}\newline
\newline
I feel that my goal to learn more about signals and systems as well as well as enhanced my overall knowledge and usage of Python to solve and understand different types of system.  
\newline
\section{Conclusion}
The noisy signal received from the aircraft positioning system, is passed through a created RLC bandpass filter which provides a clean usable output signal. 

\section{Bibliography}
\textit{CircuitLab},  \url{https://www.circuitlab.com/editor/#?id=7pq5wm&amp;from=homepage}.

\newpage


\end{document}

