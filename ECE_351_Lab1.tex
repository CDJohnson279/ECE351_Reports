%%%%%%%%%%%%%%%%%%%%%%%%%%%%%%%%%%%%%%%%%%%%%%%%%%%%%%%%%%%%%%%%
%                                                              %
% Dean Johnson                                                 %
% ECE351                                                       %
% Lab 1                                                        %
% 1/25/2022                                                    %
%                                                              %
%%%%%%%%%%%%%%%%%%%%%%%%%%%%%%%%%%%%%%%%%%%%%%%%%%%%%%%%%%%%%%%%

\documentclass[12pt]{report}
\usepackage[english]{babel}
%\usepackage{natbib}
\usepackage{url}
\usepackage[utf8x]{inputenc}
\usepackage{amsmath}
\usepackage{graphicx}
\graphicspath{{images/}}
\usepackage{parskip}
\usepackage{fancyhdr}
\usepackage{vmargin}
\usepackage{listings}
\usepackage{hyperref}
\usepackage{xcolor}
\definecolor{codegreen}{rgb}{0,0.6,0}
\definecolor{codegray}{rgb}{0.5,0.5,0.5}
\definecolor{codeblue}{rgb}{0,0,0.95}
\definecolor{backcolour}{rgb}{0.95,0.95,0.92}
\lstdefinestyle{mystyle}{
backgroundcolor=\color{backcolour},
commentstyle=\color{codegreen},
keywordstyle=\color{codeblue},
numberstyle=\tiny\color{codegray},
stringstyle=\color{codegreen},
basicstyle=\ttfamily\footnotesize,
breakatwhitespace=false,
breaklines=true,
captionpos=b,
keepspaces=true,
numbers=left,
numbersep=5pt,
showspaces=false,
showstringspaces=false,
showtabs=false,
tabsize=2
}
\lstset{style=mystyle}
\setmarginsrb{3 cm}{2.5 cm}{3 cm}{2.5 cm}{1 cm}{1.5 cm}{1 cm}{1.5 cm}
\title{1}
% Title
\author{ DEAN JOHNSON}
% Author
\date{1/18/2022}
% Date
\makeatletter
\let\thetitle\@title
\let\theauthor\@author
\let\thedate\@date
\makeatother
\pagestyle{fancy}
\fancyhf{}
\rhead{\theauthor}
\lhead{\thetitle}
\cfoot{\thepage}
%%%%%%%%%%%%%%%%%%%%%%%%%%%%%%%%%%%%%%%%%%%%
\begin{document}
%%%%%%%%%%%%%%%%%%%%%%%%%%%%%%%%%%%%%%%%%%%%%%%%%%%%%%%%%%%%%%%%%%%%%%%%%%
%%%%%%%%%%%%%%%
\begin{titlepage}
\centering
\vspace*{0.5 cm}
% \includegraphics[scale = 0.075]{bsulogo.png}\\[1.0 cm] % 
University of Idaho
\begin{center}    \textsc{\Large   ECE 351 - Section \#53 }\\[2.0 cm]
\end{center}% University Name
\textsc{\Large Introduction to Python & Latex  }\\[0.5 cm] % Course 
\rule{\linewidth}{0.2 mm} \\[0.4 cm]
{ \huge \bfseries \thetitle}\\
\rule{\linewidth}{0.2 mm} \\[1.5 cm]
\begin{minipage}{0.4\textwidth}
\begin{flushleft} \large
% \emph{Submitted To:}\\
% Name\\
% Affiliation\\
%contact info\\
\end{flushleft}
\end{minipage}~
\begin{minipage}{0.4\textwidth}
\begin{flushright} \large
\emph{Submitted By :} \\
Dean Johnson
\end{flushright}
\end{minipage}\\[2 cm]
% \includegraphics[scale = 0.5]{PICMathLogo.png}
\end{titlepage}
%%%%%%%%%%%%%%%%%%%%%%%%%%%%%%%%%%%%%%%%%%%%%%%%%%%%%%%%%%%%%%%%%%%%%%%%%%
%%%%%%%%%%%%%%%
\tableofcontents
\pagebreak
%%%%%%%%%%%%%%%%%%%%%%%%%%%%%%%%%%%%%%%%%%%%%%%%%%%%%%%%%%%%%%%%%%%%%%%%%%
%%%%%%%%%%%%%%%
\renewcommand{\thesection}{\arabic{section}}
\section{Introduction}
In each section below the parts of the handout for Lab 1 will be summarized. My github link for this class is:  \href{https://github.com/CDJohnson279}{Github}. }

\section{Part 1}
Part 1 instructs us to read over the Spyder keyboard shortcuts provided on Canvas. and then create a new file in Spyder and save it with the provided formatting. 
\section{Part 2}
Part 2 introduces the methods of defining variables, arrays, and matrices in python. Python unlike C, and C++ does not require a variable type to be specified. Other operations can be performed within a print() statement. and as with C new lines can be printed using '/n'. The handout then instructs for multiple strings and arrays to be constructed to show how Python functions. Part 3 also goes into depth on how to use multiple other Python commands such as creating sample plots.  
\section{Part 3}
Part 3 discusses proper pep8 coding practices, and provides examples on proper code. Such proper practices include the following, 4 spaces per indentations, no tabs, use docstrings to define functions, wrap lines so that they are under 79 characters use regular and updated comments, use spaces around operators and after commas.
This section also discuss proper naming conventions, and provides additional detail via the pep8 coding practices online. 
\section{Part 4}
Part 4 provides guidance on using \LaTeX  commands. In this section we are instructed to read through the Latex cheat sheets provided. Then it also provides a sample code for this lab report, however I have chosen the current formatting.  
\section{Questions}
1. Which course are you most excited for in your degree? Which course have you enjoyed the most so far?
I would say that I am most excited for ECE420 since I have found power to be quite interesting and fun. I have enjoyed Mechanics of Materials the most of all the courses I have taken. 
2. Leave any feedback on the clarity of the expectations, instructions, and deliverables.
No feedback as of now. 
\end{document}