%%%%%%%%%%%%%%%%%%%%%%%%%%%%%%%%%%%%%%%%%%%%%%%%%%%%%%%%%%%%%%%%
%                                                              %
% Dean Johnson                                                 %
% ECE351                                                       %
% Lab 9                                                    %
% 3/22/2022                                                    %
%                                                              %
%%%%%%%%%%%%%%%%%%%%%%%%%%%%%%%%%%%%%%%%%%%%%%%%%%%%%%%%%%%%%%%%

%%%%%%%%%%%%%%%%%%%%%%%%%%%%%%%%%%%%%%%%%%%
%%% DOCUMENT PREAMBLE %%%
\documentclass[12pt]{report}
\usepackage[english]{babel}
%\usepackage{natbib}
\usepackage{url}
\usepackage[utf8x]{inputenc}
\usepackage{amsmath}
\usepackage[utf8]{inputenc}
\usepackage{graphicx}
\graphicspath{{images/}}
\usepackage{parskip}
\usepackage{fancyhdr}
\usepackage{vmargin}
\usepackage{listings}
\usepackage{hyperref}
\usepackage{xcolor}

\definecolor{codegreen}{rgb}{0,0.6,0}
\definecolor{codegray}{rgb}{0.5,0.5,0.5}
\definecolor{codeblue}{rgb}{0,0,0.95}
\definecolor{backcolour}{rgb}{0.95,0.95,0.92}

\lstdefinestyle{mystyle}{
    backgroundcolor=\color{backcolour},   
    commentstyle=\color{codegreen},
    keywordstyle=\color{codeblue},
    numberstyle=\tiny\color{codegray},
    stringstyle=\color{codegreen},
    basicstyle=\ttfamily\footnotesize,
    breakatwhitespace=false,         
    breaklines=true,                 
    captionpos=b,                    
    keepspaces=true,                 
    numbers=left,                    
    numbersep=5pt,                  
    showspaces=false,                
    showstringspaces=false,
    showtabs=false,                  
    tabsize=2
}
 
\lstset{style=mystyle}

\setmarginsrb{3 cm}{2.5 cm}{3 cm}{2.5 cm}{1 cm}{1.5 cm}{1 cm}{1.5 cm}

\title{Lab 9}								
% Title
\author{ Dean Johnson}						
% Author
\date{3/22/2022}
% Date

\makeatletter
\let\thetitle\@title
\let\theauthor\@author
\let\thedate\@date
\makeatother

\pagestyle{fancy}
\fancyhf{}
\rhead{\theauthor}
\lhead{\thetitle}
\cfoot{\thepage}
%%%%%%%%%%%%%%%%%%%%%%%%%%%%%%%%%%%%%%%%%%%%
\begin{document}

%%%%%%%%%%%%%%%%%%%%%%%%%%%%%%%%%%%%%%%%%%%%%%%%%%%%%%%%%%%%%%%%%%%%%%%%%%%%%%%%%%%%%%%%%

\begin{titlepage}
	\centering
    \vspace*{0.5 cm}
   % \includegraphics[scale = 0.075]{bsulogo.png}\\[1.0 cm]	% University Logo
\begin{center}    \textsc{\Large   ECE 351 - Section \#53 }\\[2.0 cm]	\end{center}% University Name
	\textsc{\Large Fast Fourier Transform}\\[0.5 cm]				% Course Code
	\rule{\linewidth}{0.2 mm} \\[0.4 cm]
	{ \huge \bfseries \thetitle}\\
	\rule{\linewidth}{0.2 mm} \\[1.5 cm]
	
	\begin{minipage}{0.4\textwidth}
		\begin{flushleft} \large
		%	\emph{Submitted To:}\\
		%	Name\\
          % Affiliation\\
           %contact info\\
			\end{flushleft}
			\end{minipage}~
			\begin{minipage}{0.4\textwidth}
            
			\begin{flushright} \large
			\emph{Submitted By :} \\
			Dean Johnson  
		\end{flushright}
           
	\end{minipage}\\[2 cm]
	
%	\includegraphics[scale = 0.5]{PICMathLogo.png}
    
    
    
    
	
\end{titlepage}

%%%%%%%%%%%%%%%%%%%%%%%%%%%%%%%%%%%%%%%%%%%%%%%%%%%%%%%%%%%%%%%%%%%%%%%%%%%%%%%%%%%%%%%%%

\tableofcontents
\pagebreak

%%%%%%%%%%%%%%%%%%%%%%%%%%%%%%%%%%%%%%%%%%%%%%%%%%%%%%%%%%%%%%%%%%%%%%%%%%%%%%%%%%%%%%%%%
\renewcommand{\thesection}{\arabic{section}}
\section{Introduction}
The goal of this lab is to become aquainted with using Python to perform fast Fourier transforms. 
\newline \newline The GitHub link: \href{https://github.com/CDJohnson279}{Github Lab 9}. 

\section{Methodology}
To begin the lab a Fast Fourier Transform (FFT) routine is created using the outline provided in the lab handout. With this user defined fast fourier transform function we are asked to plot three different functions using this routine. For each of the provided equations we are asked to generate a single figure with 5 subplots for each. The subplots are to represent the original function, the fft of the magnitude, the fft of the magnitude zoomed in, the fft of the angle, and the fft of the angle zoomed in. Each of the equations for the first three tasks is provided below in the \textbf{Equations} section. Tasks 1,2, and 3 are at this point difficult to interpret so the FFT function is modified so that all elements of $Xmag < 1e-10$, have the $Xphi$ value set to zero. the code is re-ran and the plots for this part are included below in the \textbf{Results} section. Task 5 asks us to run the Fourier series approximation for the square wave created in the previous lab, simply for the N=15 case, this is ran through our 'clean' fft function from Task 4. The boundaries are set to $0\leq t\leq 16s$ and the period is the same as from Lab 8. 

\section{Equations}
\textbf{Task 1}\newline\newline
$cos(2 \pi t)$\newline \newline
\textbf{Task 2}\newline\newline
$5sin(2 \pi t)$\newline\newline
\textbf{Task 3} \newline\newline
$2cos((2 \pi · 2t) − 2) + sin^2
((2 \pi · 6t) + 3)$

For each of these functions the range is $0 \leq t \leq 2s$ with a sampling frequency $fs = 100.$ \newpage

\section{Results}
Plots from Task 1.\newline 
\begin{figure}[h!]
    \centering
    \includegraphics[width=10cm]{Task 1.png}
    \caption{Task 1 }
    \label{Task 1}
\end{figure}

Plots from Task 2.\newline 
\begin{figure}[h!]
    \centering
    \includegraphics[width=10cm]{Task 2.png}
    \caption{Task 2}
    \label{Task 2}
\end{figure}

Plots from Task 3.\newline 
\begin{figure}[h!]
    \centering
    \includegraphics[width=10cm]{Task 3.png}
    \caption{Task 3 }
    \label{Task 3}
\end{figure}

Plots from Task 4\newline 
\begin{figure}[h!]
    \centering
    \includegraphics[width=10cm]{Task4.1.png}
    \caption{Clean Task 1 }
    \label{Task 4.1}
\end{figure}
\newpage
\begin{figure}[h!]
    \centering
    \includegraphics[width=10cm]{Task4.2.png}
    \caption{Clean Task 2 }
    \label{Task 4.2}
\end{figure}
\begin{figure}[h!]
    \centering
    \includegraphics[width=10cm]{Task4.3.png}
    \caption{Clean Task 3 }
    \label{Task 4.3}
\end{figure}
\newpage 
Plots from Task 5 Square Wave.
\begin{figure}[h!]
    \centering
    \includegraphics[width=10cm]{Task5.png}
    \caption{Task 5 }
    \label{Task 5}
\end{figure}

\section{Questions}
\textbf{1. What happens if fs is lower? If it is higher? fs in your report must span a few orders of
magnitude.}
\newline
If the value of fs is lower, the plots become tighter more compressed. this implies a smaller frequency leads to the output being less frequent, the opposite occurs if fs increases. \newline
\textbf{2. What difference does eliminating the small phase magnitudes make?
}
\newline
By elimination the small phase magnitudes the irrelevant values are removed leaving only the critical values, making the the plots easier to interpret. 
\newline
\textbf{Verify your results from Tasks 1 and 2 using the Fourier transforms of cosine and sine.
Explain why your results are correct.}\newline
$cos(2 \pi t) \rightarrow (\pi/2)[\delta(Hz + 1) \delta(Hz-1)]$ \newline
The transform of the equation from Task 1 as shown in the plot shows two delta functions at 1 and -1, this matches up with the derived equation. 
\newline
$5sin(2 \pi t) \rightarrow (5j\pi/2)[\delta(Hz+1)-\delta(Hz-1)]$\newline
The transform of the equation from Task 2 also has 2 delta functions one at -1 and +1 and this can be verified from the graph for task 2.
\section{Conclusion}
This lab was quite useful in performing Fourier transforms fast using the fast Fourier transform function which allows us to see many aspects of the Fourier transforms that would be difficult to visualize without the computer aide. 
\newpage


\end{document}

