%%%%%%%%%%%%%%%%%%%%%%%%%%%%%%%%%%%%%%%%%%%%%%%%%%%%%%%%%%%%%%%%
%                                                              %
% Dean Johnson                                                 %
% ECE351                                                       %
% Lab 11                                                %
% 4/5/2022                                                    %
%                                                              %
%%%%%%%%%%%%%%%%%%%%%%%%%%%%%%%%%%%%%%%%%%%%%%%%%%%%%%%%%%%%%%%%

%%%%%%%%%%%%%%%%%%%%%%%%%%%%%%%%%%%%%%%%%%%
%%% DOCUMENT PREAMBLE %%%
\documentclass[12pt]{report}
\usepackage[english]{babel}
%\usepackage{natbib}
\usepackage{url}
\usepackage[utf8x]{inputenc}
\usepackage{amsmath}
\usepackage[utf8]{inputenc}
\usepackage{graphicx}
\graphicspath{{images/}}
\usepackage{parskip}
\usepackage{fancyhdr}
\usepackage{vmargin}
\usepackage{listings}
\usepackage{hyperref}
\usepackage{xcolor}
\usepackage{float}

\definecolor{codegreen}{rgb}{0,0.6,0}
\definecolor{codegray}{rgb}{0.5,0.5,0.5}
\definecolor{codeblue}{rgb}{0,0,0.95}
\definecolor{backcolour}{rgb}{0.95,0.95,0.92}

\lstdefinestyle{mystyle}{
    backgroundcolor=\color{backcolour},   
    commentstyle=\color{codegreen},
    keywordstyle=\color{codeblue},
    numberstyle=\tiny\color{codegray},
    stringstyle=\color{codegreen},
    basicstyle=\ttfamily\footnotesize,
    breakatwhitespace=false,         
    breaklines=true,                 
    captionpos=b,                    
    keepspaces=true,                 
    numbers=left,                    
    numbersep=5pt,                  
    showspaces=false,                
    showstringspaces=false,
    showtabs=false,                  
    tabsize=2
}
 
\lstset{style=mystyle}

\setmarginsrb{3 cm}{2.5 cm}{3 cm}{2.5 cm}{1 cm}{1.5 cm}{1 cm}{1.5 cm}

\title{Lab 11}								
% Title
\author{ Dean Johnson}						
% Author
\date{4/5/2022}
% Date

\makeatletter
\let\thetitle\@title
\let\theauthor\@author
\let\thedate\@date
\makeatother

\pagestyle{fancy}
\fancyhf{}
\rhead{\theauthor}
\lhead{\thetitle}
\cfoot{\thepage}
%%%%%%%%%%%%%%%%%%%%%%%%%%%%%%%%%%%%%%%%%%%%
\begin{document}

%%%%%%%%%%%%%%%%%%%%%%%%%%%%%%%%%%%%%%%%%%%%%%%%%%%%%%%%%%%%%%%%%%%%%%%%%%%%%%%%%%%%%%%%%

\begin{titlepage}
	\centering
    \vspace*{0.5 cm}
   % \includegraphics[scale = 0.075]{bsulogo.png}\\[1.0 cm]	% University Logo
\begin{center}    \textsc{\Large   ECE 351 - Section \#53 }\\[2.0 cm]	\end{center}% University Name
	\textsc{\Large Z - Transform Operations}\\[0.5 cm]				% Course Code
	\rule{\linewidth}{0.2 mm} \\[0.4 cm]
	{ \huge \bfseries \thetitle}\\
	\rule{\linewidth}{0.2 mm} \\[1.5 cm]
	
	\begin{minipage}{0.4\textwidth}
		\begin{flushleft} \large
		%	\emph{Submitted To:}\\
		%	Name\\
          % Affiliation\\
           %contact info\\
			\end{flushleft}
			\end{minipage}~
			\begin{minipage}{0.4\textwidth}
            
			\begin{flushright} \large
			\emph{Submitted By :} \\
			Dean Johnson  
		\end{flushright}
           
	\end{minipage}\\[2 cm]
	
%	\includegraphics[scale = 0.5]{PICMathLogo.png}
    
    
    
    
	
\end{titlepage}

%%%%%%%%%%%%%%%%%%%%%%%%%%%%%%%%%%%%%%%%%%%%%%%%%%%%%%%%%%%%%%%%%%%%%%%%%%%%%%%%%%%%%%%%%

\tableofcontents
\pagebreak

%%%%%%%%%%%%%%%%%%%%%%%%%%%%%%%%%%%%%%%%%%%%%%%%%%%%%%%%%%%%%%%%%%%%%%%%%%%%%%%%%%%%%%%%%
\renewcommand{\thesection}{\arabic{section}}
\section{Introduction}
The goal of this lab is to use Python functions and a provided function to analyze a discrete function in the z-domain.
\newline \newline The GitHub link: \href{https://github.com/CDJohnson279}{Github Lab 11}. 

\section{Methodology}
For this lab we are given a causal function \textbf{ y[k] = 2x[k] − 40x[k − 1] + 10y[k − 1] − 16y[k − 2]} by hand we determine the transfer function in the Z domain and h[k] by partial fraction expansion, these are shown below in the \textbf{Equation} section. Then using the scipy residuez function we verify the partial fraction expansion, the output from this task is included in the \textbf{Appendix} section at the end of this report. using the provided zplane function we plot the pole-zero plot for the derived H(z) function, and lastly the sci
freqz function is used to plot the magnitude and phase of H(z) these two plots are inlcuded below in the \textbf{Results} section.
\section{Equations}
Task 1:
$H(z) = (2z(z-20))/(z-20)(z-8)$ \newline
Task 2:
$H[k] = 6*2^k -4*8^k$
\newpage
\section{Results}
Plots from Part 1.\newline 

\begin{figure}[h!]
    \centering
    \includegraphics[width=10cm]{Part1.4 pole zero plot.png}
    \caption{Task 4: Pole Zero Plot for H(z) }
    \label{Task 4}
\end{figure}

\begin{figure}[h!]
    \centering
    \includegraphics[width=9cm]{Part1.5 mag and phase.png}
    \caption{Task 5:Magnitude and Phase Plots }
    \label{Task 5}
    \newpage 
\end{figure}
\newpage 


 
\section{Questions}
\textbf{1. Looking at the plot generated in Task 4, is H(z) stable? Explain why or why not.}\NEWLINE \NEWLINE
H(z) is unstable since it it goes to infinity in both directions and in the equation A(p_1)^k + B(p_2)^k, p_1 and p_2 are both greater then one so the function goes to infinity. 

\section{Conclusion}

\section{Appendix}
\begin{figure}[h!]
    \centering
    \includegraphics[width=9cm]{Part1.5 mag and phase.png}
    \caption{Task 3: Residual Results }
    \label{Task 5}
    \newpage 
\end{figure}
\newpage

\newpage


\end{document}

